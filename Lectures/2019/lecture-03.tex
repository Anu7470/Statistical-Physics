% !TEX spellcheck = en_US
% !TEX spellcheck = LaTeX
\documentclass[letterpaper,10pt,english]{article}
\usepackage{%
	amsfonts,%
	amsmath,%	
	amssymb,%
	amsthm,%
	babel,%
	bbm,%
	%biblatex,%
	caption,%
	centernot,%
	color,%
	enumerate,%
	%enumitem,%
	epsfig,%
	epstopdf,%
	etex,%
	fancybox,%
	framed,%
	fullpage,%
	%geometry,%
	graphicx,%
	hyperref,%
	latexsym,%
	mathptmx,%
	mathtools,%
	multicol,%
	pgf,%
	pgfplots,%
	pgfplotstable,%
	pgfpages,%
	proof,%
	psfrag,%
	%subfigure,%	
	tikz,%
	times,%
	ulem,%
	url,%
	xcolor,%
	mathpazo
}

\definecolor{shadecolor}{gray}{.95}%{rgb}{1,0,0}
\usepackage[margin=1in,top=0.75in]{geometry}
\usepackage[mathscr]{eucal}
\usepgflibrary{shapes}
\usepgfplotslibrary{fillbetween}
\usetikzlibrary{%
  arrows,%
  backgrounds,%
  chains,%
  decorations.pathmorphing,% /pgf/decoration/random steps | erste Graphik
  decorations.text,% 
  matrix,%
  positioning,% wg. " of "
  fit,%
  patterns,%
  petri,%
  plotmarks,%
  scopes,%
  shadows,%
  shapes.misc,% wg. rounded rectangle
  shapes.arrows,%
  shapes.callouts,%
  shapes%
}

%\pgfplotsset{compat=newest} %<------ Here
\pgfplotsset{compat=1.11} %<------ Or use this one

\theoremstyle{plain}
\newtheorem{thm}{Theorem}[section]
\newtheorem{lem}[thm]{Lemma}
\newtheorem{prop}[thm]{Proposition}
\newtheorem{cor}[thm]{Corollary}
\newtheorem{clm}[thm]{Claim}

\theoremstyle{definition}
\newtheorem{axiom}[thm]{Axiom}
\newtheorem{defn}[thm]{Definition}
\newtheorem{conj}[thm]{Conjecture}
\newtheorem{exmp}[thm]{Example}
\newtheorem{assum}[thm]{Assumptions}

\theoremstyle{remark}
\newtheorem{rem}[thm]{Remark}
\newtheorem{note}[thm]{Note}

\newcommand{\Cov}{\operatorname{Cov}}
%\newcommand{\det}{\operatorname{det}}
\newcommand{\Real}{\mathbb{R}}
\newcommand{\tr}{\operatorname{tr}}
%\newcommand{\Var}{\operatorname{Var}}

%\renewcommand{\proof}[1]{\begin{proof}#1\end{proof}}
\newcommand{\EQ}[1]{\begin{equation*}#1\end{equation*}}
\newcommand{\EQN}[1]{\begin{equation}#1\end{equation}}
\newcommand{\eq}[1]{\begin{align*}#1\end{align*}}
\newcommand{\meq}[2]{\begin{xalignat*}{#1}#2\end{xalignat*}}
\newcommand{\norm}[1]{\left\lVert#1\right\rVert}
\newcommand{\abs}[1]{\left\lvert#1\right\rvert}
\newcommand{\expect}[1]{\mathbb{E}\left[{#1}\right]}
\newcommand{\prob}[1]{\mathbb{P}\left[{#1}\right]}
\newcommand{\given}{\; \big\vert \;} 
\newcommand{\set}[1]{\left\{#1\right\}} 
\newcommand{\indicator}[1]{\mathbb{1}_{\set{#1}}} 
\newcommand{\red}[1]{\textcolor{red}{#1}} 
\newcommand{\E}[1]{\mathbb{E}\left[#1\right]}
\newcommand{\Var}[1]{\operatorname{Var}\left[#1\right]}

\newcommand{\D}{\mathbb{D}}
%\newcommand{\E}{\mathbb{E}}
\newcommand{\N}{\mathbb{N}}
\renewcommand{\P}{\mathbb{P}}
\newcommand{\Q}{\mathbb{Q}}
\newcommand{\R}{\mathbb{R}}
\newcommand{\Z}{\mathbb{Z}}

\newcommand{\bU}{\mathbf{1}}
\newcommand{\bx}{\mathbf{x}}

\newcommand{\cB}{\mathcal{B}}
\newcommand{\cC}{\mathcal{C}}
\newcommand{\cD}{\mathcal{D}}
\newcommand{\cF}{\mathcal{F}}
\newcommand{\cG}{\mathcal{G}}
\newcommand{\cH}{\mathcal{H}}
\newcommand{\cT}{\mathcal{T}}
\newcommand{\cX}{\mathcal{X}}
\newcommand{\cY}{\mathcal{Y}}

\newcommand{\sA}{\mathscr{A}}
\newcommand{\sB}{\mathscr{B}}
\newcommand{\sC}{\mathscr{C}}
\newcommand{\sD}{\mathscr{D}}
\newcommand{\sE}{\mathscr{E}}
\newcommand{\sF}{\mathscr{F}}
\newcommand{\sG}{\mathscr{G}}
\newcommand{\sH}{\mathscr{H}}
\newcommand{\sL}{\mathscr{L}}
\newcommand{\sS}{\mathscr{S}}
\newcommand{\sT}{\mathscr{T}}
\newcommand{\sX}{\mathscr{X}}
\newcommand{\sY}{\mathscr{Y}}

% Debug
\newcommand{\todo}[1]{\begin{color}{blue}{{\bf~[TODO:~#1]}}\end{color}}

% a few handy macros

\renewcommand{\le}{\leqslant}
\renewcommand{\ge}{\geqslant}
\newcommand\matlab{{\sc matlab}}
\newcommand{\goto}{\rightarrow}
\newcommand{\bigo}{{\mathcal O}}
%\newcommand{\half}{\frac{1}{2}}
%\newcommand\implies{\quad\Longrightarrow\quad}
\newcommand\reals{{{\rm l} \kern -.15em {\rm R} }}
\newcommand\complex{{\raisebox{.043ex}{\rule{0.07em}{1.56ex}} \hskip -.35em {\rm C}}}


% macros for matrices/vectors:

% matrix environment for vectors or matrices where elements are centered
\newenvironment{mat}{\left[\begin{array}{ccccccccccccccc}}{\end{array}\right]}
\newcommand\bcm{\begin{mat}}
\newcommand\ecm{\end{mat}}

% matrix environment for vectors or matrices where elements are right justifvied
\newenvironment{rmat}{\left[\begin{array}{rrrrrrrrrrrrr}}{\end{array}\right]}
\newcommand\brm{\begin{rmat}}
\newcommand\erm{\end{rmat}}

% for left brace and a set of choices
%\newenvironment{choices}{\left\{ \begin{array}{ll}}{\end{array}\right.}
\newcommand\when{&\text{if~}}
\newcommand\otherwise{&\text{otherwise}}
% sample usage:
%  \delta_{ij} = \begin{choices} 1 \when i=j, \\ 0 \otherwise \end{choices}


% for labeling and referencing equations:
\newcommand{\eql}{\begin{equation}\label}
\newcommand{\eqn}[1]{(\ref{#1})}
% can then do
%  \eql{eqnlabel}
%  ...
%  \end{equation}
% and refer to it as equation \eqn{eqnlabel}.  


% some useful macros for finite difference methods:
\newcommand\unp{U^{n+1}}
\newcommand\unm{U^{n-1}}

% for chemical reactions:
\newcommand{\react}[1]{\stackrel{K_{#1}}{\rightarrow}}
\newcommand{\reactb}[2]{\stackrel{K_{#1}}{~\stackrel{\rightleftharpoons}
   {\scriptstyle K_{#2}}}~}


\makeatletter
\def\th@plain{%
  \thm@notefont{}% same as heading font
  \itshape % body font
}
\def\th@definition{%
  \thm@notefont{}% same as heading font
  \normalfont % body font
}
\makeatother
\date{}

\title{Lecture-03: Data Processing}
\author{}

\begin{document}
\maketitle

\section{Data Processing}
The definitions of entropy, mutual information, and divergence all extend naturally to any finite number of random variables by treating multiple random variables as a single random vector. However, there are a few new concepts that can only be defined in terms of three random variables. Let $X, Y$, and $Z$ be random variables with joint distribution $p_{X,Y,Z}(x,y,z)$. 
\begin{defn} 
For three r.v. $(X,Y,Z) \sim p_{X,Y,Z}(x,y,z)$ defined on $\sX \times \sY \times \sZ$, 
the conditional mutual information (in bits) between $X$ and $Y$ given $Z$ is denoted 
\EQ{
I(X;Y | Z)  \triangleq \sum_{(x,y,z) \in \sX \times \sY \times \sZ}p_{X,Y,Z}(x,y,z)\log_2\frac{p_{X,Y|Z}(x,y,z)}{p_{X|Z}(x,z)p_{Y|Z}(y,z)} = \E{\log_2\frac{p_{X,Y|Z}(X,Y,Z)}{p_{X|Z}(X,Z)p_{Y|Z}(Y,Z)}}.
}
From this, we see that $I(X;Y | Z) = H(X | Z) + H(Y|Z) - H(X,Y |Z)$. 
Thus the conditioning is simply inherited by each entropy in the standard decomposition. 
\end{defn} 
\begin{defn} 
Three r.v. $(X, Y, Z) \sim p_{X,Y, Z} (x, y, z)$ form a Markov chain $X -Y - Z$ if 
\EQ{
p_{X,Y,Z} (x, y, z) = p_X (x)p_{Y |X} (y|x)p_{Z|Y}(z|y).
}
This is clearly the same as $p_{Z|X,Y} (z|x, y) = p_{Z|Y} (z | y)$ for all $x, y, z$, 
which is equivalent to the condition that $X$ and $Z$ are conditionally independent given $Y$. 
\end{defn} 

\begin{lem}
Properties of mutual information for three random variables:
\begin{enumerate}
\item (chain rule of mutual information) $I(X;Y,Z)=I(X;Y)+I(X;Z|Y)$. 
\begin{proof}
This follows from the expectation of the decomposition
\EQ{
\log_2\frac{p_{X,Y,Z}(X,Y,Z)}{p_X(x)p_{Y,Z}(Y,Z)}= \log_2\frac{p_{X,Y}(X,Y)p_{Z|X,Y}(Z|X,Y)}{p_X(x)p_Y(Y)p_{Z|Y}(Z|Y)}
= \log_2\frac{p_{X,Y}(X,Y)}{p_X(x)p_Y(Y)} + \log_2\frac{p_{X,Z |Y}(X,Z|Y)}{p_{Z|Y}(Z|Y)p_{X|Y}(X|Y)}.
}
\end{proof} 
\item (non-negativity of conditional mutual information) $I (X ; Y |Z ) \ge 0$ with equality iff $X$ and $Y$ are conditionally independent given $Z$.
\begin{proof}
First, we observe that
\EQ{
I (X; Y |Z) = \sum_z p_Z(z)D(p_{X,Y |Z=z}\Vert p_{X|Z=z}p_{Y |Z=z}).
} 
Each term in this sum is non-negative and equal to zero iff $p_{X,Y|Z=z}(x,y) = p_{X|Z=z}(x)p_{Y|Z=z}(y)$ for all $x,y$. 
Thus, the overall sum is zero iff the condition holds for all $x, y, z$ (i.e., $X$ and $Y$ are conditionally independent given $Z$). 
\end{proof} 
\end{enumerate}
\end{lem} 
    
\begin{thm}[Data Processing Inequality] 
If three r.v. $(X, Y, Z) \sim p_{X,Y,Z} (x, y, z)$ form a Markov chain $X - Y- Z$, then 
$I(X;Z) \le I(X;Y)$. 
For example, if $Z=f(Y)$ is a function of $Y$, 
then $X -Y - Z$ form a Markov chain. 
\end{thm} 
\begin{proof}
Applying the chain rule of mutual information in the two possible orders gives 
\EQ{
I (X; Y, Z) = I (X; Z) + I (X; Y |Z) = I (X; Y) + I (X; Z|Y).
}
Since $X - Y - Z$ form a Markov chain, $X$ and $Z$ are conditionally independent and 
$I(X;Z|Y) = 0$. 
Thus, we have 
\EQ{
I (X; Y ) = I (X; Z) + I (X; Y |Z) \ge I (X; Z).
}
If $Z = f(Y)$, then $p_{Z|X,Y}(z|x,y) = \indicator{z =f(y)} = p_{Z|Y}(z,y)$ and $X - Y - Z$ form a Markov chain.
\end{proof} 
\begin{shaded*}\begin{exmp}
A system has a random state $X$ and an experiment with outcome $Y$ is performed to measure that state. 
Is it possible that additional processing can produce a new output $Z = f(Y)$ such that 
$H (X|Z) < H (X|Y )$?
\end{exmp}\end{shaded*}
\begin{thm} [Fano's Inequality] 
Let the r.v. $Y$ be an observation of the r.v. $X$ and $\hat{X} = f(Y)$ be an estimate of $X$. 
Then, the error probability $P_e = P(\hat{X} \neq X)$ satisfies 
\EQ{
\cH(P_e) + P_e \log_2 (|\sX| - 1) \ge H (X|Y).
}
\end{thm} 
\begin{proof}
Let $E = \indicator{\hat{X}= X}$ be an indicator r.v. for the error event. 
Expanding the conditional entropy $H(E, X |\hat{X})$ in two ways gives 
\EQ{
H(E, X |\hat{X}) = H(X| \hat{X}) + H(E|X, \hat{X}) = H(E| \hat{X}) + H(X|E, \hat{X}).
}
Now $H(X|\hat{X}) \ge H(X|Y)$ by data processing inequality, since $X - Y - \hat{X}$ form a Markov chain, 
and $H(E| X, \hat{X}) = 0$ since $E = \indicator{X \neq \hat{X}}$. 
Further, $H(E|\hat{X}) \le H(E) = \cH(P_e)$ since the conditioning reduces entropy. 
In addition, $H(X|E=0, \hat{X}) = 0$, and we can write 
\EQ{
H(X|E=1, \hat{X}) \le H(X \neq \hat{X}) \le \log_2(\abs{\sX}-1). 
}
This implies that $H(X|E, \hat{X}) \le P_e\log_2(\abs{\sX}-1)$. 
Rearranging these terms gives the stated result.
\end{proof}
\section{Sequences of random variables}
Let $(X_t: t \in \N)$ be a random process where each random variable lies in $\sX$. 
The joint probability distribution of the first $N$ random variables is denoted 
$P_N (x_1, \dots , x_N)$. 
Let $[N] \triangleq \set{1,2, \dots,N}, A \subseteq [N]$, and $\bar{A} = [N]\setminus A$ be sets of indices. 
We will denote subvectors with indices in $A$ and $\bar{A}$ by
\meq{2}{
&x_A =(x_t: t\in A),&&x_{\bar{A}} =(x_t: t \in \bar{A}).
}
The marginal distribution of variables in $A$ is given by summing over all variables in $\bar{A}$: \EQ{
P_A(x_A) = \sum_{x_{\bar{A}}}P_N(x_1, \dots,x_N).
}
\begin{defn}
The entropy rate of a random process is defined to be
\EQ{
h_X \triangleq \lim_{N \to \infty} \frac{1}{N}H(X_1,X_2, \dots,X_N),
}
if the limit exists.
\end{defn} 
\begin{shaded*}\begin{exmp}
If the random variables are drawn i.i.d. according to $p(x)$, then 
\EQ{
P_N(x_1, \dots, x_N) = \prod_{t=1}^Np(x_t). 
}
In this case, $H (X_1, \dots,X_N) = NH (p)$ and the entropy rate is $h_X = H (p)$.
\end{exmp}\end{shaded*} 
\begin{shaded*}\begin{exmp}
If the random variables form a homogenous Markov chain, then 
\EQ{
P_N(x_1, \dots ,x_N) = p_1(x_1)\prod_{t=1}^N w(x_t \to  x_{t+1}),
}
where $p_1(x)$ is the distribution of the initial state and $w(x \to x') = p_{X_{t+1}|X_t} (x'|x)$ defines the transition probabilities of the chain. 
In this case, the entropy rate is given by
\eq{
h_X &= \lim_{N \to \infty} \frac{1}{N}H(X_1,X_2, \dots,X_N) = \lim_{N \to \infty} \frac{1}{N}\left(H(X_1) + \sum_{t=1}^{N-1}H(X_{t+1}|X_t)\right)\\
&= \lim_{N \to \infty} \frac{1}{N}\sum_{t=1}^{N-1}\sum_{x \in \sX}p_t(x)\sum_{x' \in \sX}w(x \to x')\log_2\frac{1}{w(x \to x')}\\
&=\sum_{x \in \sX}\left(\lim_{N \to \infty} \frac{1}{N} \sum_{t=1}^{N-1}p_t(x)\right)\sum_{x' \in \sX}w(x \to x')\log_2\frac{1}{w(x \to x')}\\
&=\sum_{x \in \sX}p^{\ast}(x)\sum_{x' \in \sX}w(x \to x')\log_2\frac{1}{w(x \to x')},
}
where the last step assumes that $w(x \to x')$ was chosen so that the limiting occupancy
distribution $p^{\ast}(x) = \lim_{N \to \infty}\frac{1}{N} \sum_{t=1}^{N-1}p_t(x)$ exists and is independent of the initial state distribution. 
\end{exmp}\end{shaded*}

\end{document}
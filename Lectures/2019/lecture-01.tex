% !TEX spellcheck = en_US
% !TEX spellcheck = LaTeX
\documentclass[letterpaper,10pt,english]{article}
\usepackage{%
	amsfonts,%
	amsmath,%	
	amssymb,%
	amsthm,%
	babel,%
	bbm,%
	%biblatex,%
	caption,%
	centernot,%
	color,%
	enumerate,%
	%enumitem,%
	epsfig,%
	epstopdf,%
	etex,%
	fancybox,%
	framed,%
	fullpage,%
	%geometry,%
	graphicx,%
	hyperref,%
	latexsym,%
	mathptmx,%
	mathtools,%
	multicol,%
	pgf,%
	pgfplots,%
	pgfplotstable,%
	pgfpages,%
	proof,%
	psfrag,%
	%subfigure,%	
	tikz,%
	times,%
	ulem,%
	url,%
	xcolor,%
	mathpazo
}

\definecolor{shadecolor}{gray}{.95}%{rgb}{1,0,0}
\usepackage[margin=1in,top=0.75in]{geometry}
\usepackage[mathscr]{eucal}
\usepgflibrary{shapes}
\usepgfplotslibrary{fillbetween}
\usetikzlibrary{%
  arrows,%
  backgrounds,%
  chains,%
  decorations.pathmorphing,% /pgf/decoration/random steps | erste Graphik
  decorations.text,% 
  matrix,%
  positioning,% wg. " of "
  fit,%
  patterns,%
  petri,%
  plotmarks,%
  scopes,%
  shadows,%
  shapes.misc,% wg. rounded rectangle
  shapes.arrows,%
  shapes.callouts,%
  shapes%
}

%\pgfplotsset{compat=newest} %<------ Here
\pgfplotsset{compat=1.11} %<------ Or use this one

\theoremstyle{plain}
\newtheorem{thm}{Theorem}[section]
\newtheorem{lem}[thm]{Lemma}
\newtheorem{prop}[thm]{Proposition}
\newtheorem{cor}[thm]{Corollary}
\newtheorem{clm}[thm]{Claim}

\theoremstyle{definition}
\newtheorem{axiom}[thm]{Axiom}
\newtheorem{defn}[thm]{Definition}
\newtheorem{conj}[thm]{Conjecture}
\newtheorem{exmp}[thm]{Example}
\newtheorem{assum}[thm]{Assumptions}

\theoremstyle{remark}
\newtheorem{rem}[thm]{Remark}
\newtheorem{note}[thm]{Note}

\newcommand{\Cov}{\operatorname{Cov}}
%\newcommand{\det}{\operatorname{det}}
\newcommand{\Real}{\mathbb{R}}
\newcommand{\tr}{\operatorname{tr}}
%\newcommand{\Var}{\operatorname{Var}}

%\renewcommand{\proof}[1]{\begin{proof}#1\end{proof}}
\newcommand{\EQ}[1]{\begin{equation*}#1\end{equation*}}
\newcommand{\EQN}[1]{\begin{equation}#1\end{equation}}
\newcommand{\eq}[1]{\begin{align*}#1\end{align*}}
\newcommand{\meq}[2]{\begin{xalignat*}{#1}#2\end{xalignat*}}
\newcommand{\norm}[1]{\left\lVert#1\right\rVert}
\newcommand{\abs}[1]{\left\lvert#1\right\rvert}
\newcommand{\expect}[1]{\mathbb{E}\left[{#1}\right]}
\newcommand{\prob}[1]{\mathbb{P}\left[{#1}\right]}
\newcommand{\given}{\; \big\vert \;} 
\newcommand{\set}[1]{\left\{#1\right\}} 
\newcommand{\indicator}[1]{\mathbb{1}_{\set{#1}}} 
\newcommand{\red}[1]{\textcolor{red}{#1}} 
\newcommand{\E}[1]{\mathbb{E}\left[#1\right]}
\newcommand{\Var}[1]{\operatorname{Var}\left[#1\right]}

\newcommand{\D}{\mathbb{D}}
%\newcommand{\E}{\mathbb{E}}
\newcommand{\N}{\mathbb{N}}
\renewcommand{\P}{\mathbb{P}}
\newcommand{\Q}{\mathbb{Q}}
\newcommand{\R}{\mathbb{R}}
\newcommand{\Z}{\mathbb{Z}}

\newcommand{\bU}{\mathbf{1}}
\newcommand{\bx}{\mathbf{x}}

\newcommand{\cB}{\mathcal{B}}
\newcommand{\cC}{\mathcal{C}}
\newcommand{\cD}{\mathcal{D}}
\newcommand{\cF}{\mathcal{F}}
\newcommand{\cG}{\mathcal{G}}
\newcommand{\cH}{\mathcal{H}}
\newcommand{\cT}{\mathcal{T}}
\newcommand{\cX}{\mathcal{X}}
\newcommand{\cY}{\mathcal{Y}}

\newcommand{\sA}{\mathscr{A}}
\newcommand{\sB}{\mathscr{B}}
\newcommand{\sC}{\mathscr{C}}
\newcommand{\sD}{\mathscr{D}}
\newcommand{\sE}{\mathscr{E}}
\newcommand{\sF}{\mathscr{F}}
\newcommand{\sG}{\mathscr{G}}
\newcommand{\sH}{\mathscr{H}}
\newcommand{\sL}{\mathscr{L}}
\newcommand{\sS}{\mathscr{S}}
\newcommand{\sT}{\mathscr{T}}
\newcommand{\sX}{\mathscr{X}}
\newcommand{\sY}{\mathscr{Y}}

% Debug
\newcommand{\todo}[1]{\begin{color}{blue}{{\bf~[TODO:~#1]}}\end{color}}

% a few handy macros

\renewcommand{\le}{\leqslant}
\renewcommand{\ge}{\geqslant}
\newcommand\matlab{{\sc matlab}}
\newcommand{\goto}{\rightarrow}
\newcommand{\bigo}{{\mathcal O}}
%\newcommand{\half}{\frac{1}{2}}
%\newcommand\implies{\quad\Longrightarrow\quad}
\newcommand\reals{{{\rm l} \kern -.15em {\rm R} }}
\newcommand\complex{{\raisebox{.043ex}{\rule{0.07em}{1.56ex}} \hskip -.35em {\rm C}}}


% macros for matrices/vectors:

% matrix environment for vectors or matrices where elements are centered
\newenvironment{mat}{\left[\begin{array}{ccccccccccccccc}}{\end{array}\right]}
\newcommand\bcm{\begin{mat}}
\newcommand\ecm{\end{mat}}

% matrix environment for vectors or matrices where elements are right justifvied
\newenvironment{rmat}{\left[\begin{array}{rrrrrrrrrrrrr}}{\end{array}\right]}
\newcommand\brm{\begin{rmat}}
\newcommand\erm{\end{rmat}}

% for left brace and a set of choices
%\newenvironment{choices}{\left\{ \begin{array}{ll}}{\end{array}\right.}
\newcommand\when{&\text{if~}}
\newcommand\otherwise{&\text{otherwise}}
% sample usage:
%  \delta_{ij} = \begin{choices} 1 \when i=j, \\ 0 \otherwise \end{choices}


% for labeling and referencing equations:
\newcommand{\eql}{\begin{equation}\label}
\newcommand{\eqn}[1]{(\ref{#1})}
% can then do
%  \eql{eqnlabel}
%  ...
%  \end{equation}
% and refer to it as equation \eqn{eqnlabel}.  


% some useful macros for finite difference methods:
\newcommand\unp{U^{n+1}}
\newcommand\unm{U^{n-1}}

% for chemical reactions:
\newcommand{\react}[1]{\stackrel{K_{#1}}{\rightarrow}}
\newcommand{\reactb}[2]{\stackrel{K_{#1}}{~\stackrel{\rightleftharpoons}
   {\scriptstyle K_{#2}}}~}


\makeatletter
\def\th@plain{%
  \thm@notefont{}% same as heading font
  \itshape % body font
}
\def\th@definition{%
  \thm@notefont{}% same as heading font
  \normalfont % body font
}
\makeatother
\date{}

\title{Lecture-01: Introduction to Information Theory}
\author{}

\begin{document}
\maketitle

\section{Random Variables} 

Our main focus will be on the behavior of large sets of discrete random variables. 
%From a mathematical perspective, this class deals primarily with probability and, in particular, discrete random variables. 
\begin{defn} 
A \textbf{discrete random variable}, $X$, is defined by following information: 
(i) $\sX$ : the finite set of values that it may take, 
(ii) $p_X : \sX \to [0, 1]$: the probability it takes each value $x \in X$ . 
Of course, the probability distribution $p_X$ must satisfy the normalization condition $\sum_{x \in X}p_X (x) = 1$. 
If there is no ambiguity, we may use $p(x)$ to denote $p_X (x)$. 
\end{defn}
\begin{exmp}
Let the random variable $X$ denote the sum of two fair 6-sided dice. 
Then, $\sX = \set{2, 3, \dots, 12}$ and 
\EQ{
p_X (x) = \frac{6 - \abs{7 - x}} {36}.
}
\end{exmp}
\begin{defn}
An \textbf{event} $A \subseteq \sX$ is a subset of values. 
The probability of an event is denoted 
\EQ{
\P(X \in A)=\P(A) = \sum_{x \in A}p_X(x) = \sum_{x \in A}\P(X =x).
}
Also, an event is sometimes defined in words, $A = ``X \text{ is even}''$.
\end{defn}
\begin{exmp} 
If X is the sum of two fair 6-sided dice and $A = ``X \text{ is even}''$. Then, 
\EQ{
\P(X \text{ is even}) = \P(A) = \sum_{x \in A}p_X (x) = \frac{1 + 3 + 5 + 5 + 3 + 1}{36} = \frac{1}{2}.
}
\end{exmp}
\begin{defn} 
For a discrete random variable, the expected value (or average) of $f : \sX \to \R$ is denoted
\EQ{
\E f = \E[f(X)] = \sum_{x \in \sX}p_X(x)f(x).
}
Mathematically, $\E{}$ can be seen as a linear operator from the space of real functions on $\sX$ to the set of real numbers. 
Thus,
\EQ{
\E [af(X) + bg(X)] = a\E [f(X)] + b\E [g(X)].
}
\end{defn}

\begin{exmp}
If $X$ is the sum of two fair 6-sided dice and $f(x) = (x - 7)^2$, 
then 
\EQ{
\E[(X-7)^2] =\sum_{x \in A}p_X(x)(x-7)^2 = \frac{2(1\cdot 5^2 +2\cdot 4^2 +3\cdot 3^2 +4\cdot 2^2 +5\cdot 1^2)}{36} = \frac{105}{18}.
}
Since the mean is $\E[X] = 7$, this actually equals the variance of $X$.
\end{exmp}
\begin{defn} 
A continuous random variable, $X$, taking values on the set $\sX = \R^d$ or in some smooth finite-dimensional manifold is defined by its cumulative distribution function $\P(X \le x)$, where $X \le x$ is used to denote
$X_i \le x_i$ for $i = 1, \dots, d$. For such a r.v., the probability measure with respect to the
infinitesimal element $dx$ is denoted by $dp_X (x)$. 
For a measurable event $\sA \subseteq X$ , this gives 
\EQ{
\P(X \in \sA) = \int_{\sA}dp_X(x) = \int\indicator{x \in \sA}dp_X(x),
}
where the indicator function $\indicator{s}$ is $1$ if the logical statement $s$ is true and $0$ otherwise. 
If $p_X$ admits a density, with respect to Lebesgue measure, then it will be denoted by $p_X(x)$. 
In this case, we can write
\EQ{
\P(X \in \sA) = \int_{\sA}p_X(x)dx = \int\indicator{x \in \sA}p_X(x)dx.
}
\end{defn}
\begin{exmp}
If $X$ is a continuous random variable defined, for $a, b \in \R$ with $a < b$, by
\EQ{
\P(X \le x) = 
\begin{cases}
0 &  x < a\\
\frac{x-a}{b-a} & a \le x \le b\\
1 & x > b,
\end{cases}
}
then it is uniform on $[a, b]$ and its density is given by $p_X (x) = \frac{1}{b-a}\indicator{x \in [a,b]}$.
\end{exmp}

\begin{defn} 
The expected value and variance of a function $f : \R^d \to \R$ of a continuous
random variable $X \in \R^d$ are given by
\eq{
&\E f = \E[f(X)] = \int f(x)dp_X(x),\\
&\Var{f} =\Var{f(X)}=\E[(f(X)- \E[f(X)])^2]=\E[f(X)^2]-\E[f(X)]^2.
}
\end{defn}
\begin{exmp} 
If $X$ is a continuous random variable that is uniform on $[a, b]$, 
then its mean and variance are given by
\eq{
&\E{X} = \int_a^b \frac{x}{b-a}dx = \frac{b^2-a^2}{2(b-a)} = \frac{b+a}{2},\\
&\Var{X} =  \int_a^b \frac{x^2}{b-a}dx -\left(\frac{b+a}{2}\right)^2= \frac{b^3-a^3}{3(b-a)} -\left(\frac{b+a}{2}\right)^2= \frac{(b-a)^2}{12}.
}
\end{exmp}
     
\section{Entropy}
In statistical mechanics, the entropy is proportional to the logarithm of the number of resolvable microstates associated with a macrostate. 
In classical mechanics, this quantity contains an arbitrary additive constant associated with the size of a microstate that is considered resolvable. 
In quantum mechanics, there is a natural limit to resolvability and this constant is related to the Planck constant. 
For random variables, Shannon chose the following definition which is similar in spirit. 
\begin{defn} 
The \textbf{entropy} (in bits) of a discrete random variable $X$ with probability distribution $p(x)$ is denoted
\EQ{
H(X) \triangleq -\sum_{x \in \sX}p(x)\log_2p(x)=\E{\frac{1}{\log_2p(X)}},
}
where $0 \log_2 0 = 0$ by continuity. 
The notation $H (p)$ is used to denote $H (X )$ when $X \sim p(x)$. 
When there is no ambiguity, $H$ will be used instead of $H (X)$. 
The unit of entropy is determined by the base of the logarithm with base-2 resulting in ``bits'' and the natural log (i.e., base-e) resulting in ``nats''. 
\begin{rem}
Roughly speaking, the entropy $H (X)$ measures the uncertainty in the random variable $X$.
\end{rem}
\begin{exmp} 
If $X$ is uniform, then $p(x) = \frac{1}{\abs{\sX}}$ and 
\EQ{
H (X) = \E{\log_2 \frac{1}{\abs{\sX}}} = \log_2 \abs{\sX}.
}
Choosing $\abs{\sX} = 2$, we see that a uniform random bit has exactly $\log_2 2 = 1$ bit of entropy.
\end{exmp}
\begin{exmp}
Let $X$ be a binary r.v. defined by $p(0) = 1-q$ and $p(1) = q$. 
In this case, we have
\EQ{
H(X)= \cH(q) = q\log_2 \frac{1}{q}+(1-q)\log_2\frac{1}{1-q}, 
}
where $\cH(q)$ is called the binary entropy function. 
This function is concave and symmetric about $q = \frac{1}{2}$. 
It also satisfies $\cH(0) = \cH(1) = 0$ and $\cH(1/2) = 1$.
\end{exmp}

   
                     

\begin{exmp}
The number of length-$n$ binary sequences with exactly $qn$ ones is given by $\binom{n}{qn}$. 
Using Stirling's formula, $n! = \sqrt{2\pi n}(\frac{n}{e})^n(1 + O(\frac{1}{n})$, we see that 
\eq{
\binom{n}{qn} &= \frac{n!}{(n-qn)!(qn)!}\\
&=
?2?n??n??n ??1+O??1???? en
    ????????????? ???????? ??   (n?qn) n?qn 1 ??qn??qn 1
  2?(n?qn) e 1+O n?qn 2?qn e 1+O qn 1 ??1+O??1????
     n
= ??2?nq(1?q)(1?q)n(1?q) (q)qn � ??1+O?? 1 ??????1+O?? 1 ????
n?qn qn 1 ??1??qn ?? 1 ??(1?q)n ?? ?? 1 ????
     = ??2?nq(1?q) q 1?q 1+O nq(1?q)
1 n[qlog 1+(1?q)log 1 ]?? ?? 1 ????
     =?? 22q 21?q1+O 2?nq(1 ? q)
     nq(1 ? q) = ??2?nq(1?q)2 1+O nq(1?q) .
1 nH(q)?? ?? 1 ????
}
\end{exmp}
\begin{rem} 
This shows that the binary entropy determines the exponential growth rate of the number of binary sequences with a fixed fraction of ones. 
In fact, this is a fundamental property of the entropy. 
More generally, we will see that the entropy $H (X)$ is the exponential growth rate of the number of length-$n$ sequences (i.e., there are roughly $2^{nH(X)}$ such sequences) where the fraction of $x's$ converges to $np(x)$. 
This also implies that $nH (X)$ is essentially equal to the minimum number of binary digits required to index all length-$n$ sequences of this type. 
\end{rem}
\begin{lem} 
Basic properties of entropy:
\begin{enumerate}
\item (non-negativity of entropy) H (X) ? 0 with equality iff X is constant.
H(q)

\begin{proof} 
log 1 ? 0 and, if X is not constant, there is an x with p(x ) ? (0,1). Thus,
 2 p(x)
H (X) ? p(x0) log2(1/p(x0)).
0 0
\end{proof}
2. (decomposition rule for entropy) For any partition A1 ? A2 ? � � � ? Am = X , we have m
H(p) = H(pA) + ?? p(Ai)H(pi), i=1
where we define pA(i) = p(Ai) ? ??x?Ai p(x) for i = 1,2,...,m and pi(x) = p(x)/p(Ai) for x ? Ai.
Proof: Observe that
m1 H(p) = ?? ?? p(x) log2 p(x)
i=1 x?Am
m?? p(A)??
=?? ?p(A)logp(A)+ ?? p(x)log
i
2 p(x)
p(x) 1
i=1 m
x?Am +??p(Ai) ??
=??p(Ai)log2 i=1
m
p(Ai )
log2
p(x)
i
i
1 m
5
     p(Ai )
= H(pA) + ?? p(Ai)H(pi).
i=1
\end{enumerate}
\end{lem}
Example 18. Compute the entropy of the distribution p(x) = [0.125 0.375 0.25 0.25]. Using
decomposition with A1 = {1, 2} and A2 = {3, 4}, we get
H (p) = 1 + 0.5H(1/4) + 0.5 ? 1.9056.
III. MUTUAL INFORMATION
Definition 19. The joint entropy (in bits) of a pair of r.v. (X, Y ) ? pX,Y (x, y) is denoted
??1??1??
H(X,Y)?
Notice that this is identical to H (Z) with Z = (X, Y ).
pXY(x,y)log2 pX,Y(x,y) =E log2 pX,Y(X,Y) . Definition 20. For a pair of r.v. (X, Y ) ? pX,Y (x, y), the conditional entropy (in bits)
i=1 x?Am
  of Y given X is denoted H(Y|X)?
(x,y)?X �Y
??1??1??
pX,Y(x,y)log2 pY|X(y|x) =E log2 pY|X(Y|X) .
Notice that this equals entropy of the conditional distribution pY |X (y|x) averaged over x.
  (x,y)?X �Y

H(X) H(Y)
?? pX,Y(x,y) ?? pX,Y(X,Y) ??
6
  H(X|Y ) I(X; Y ) H(Y |X)
Figure 2: Venn diagram illustrating H (X), H (Y ), H (X|Y ), H (Y |X), and I (X; Y ). Definition 21. For a pair of r.v. (X, Y ) ? pX,Y (x, y), the mutual information (in bits)
between X and Y is denoted
I(X;Y)?
Lemma 22. Basic properties of joint entropy and mutual information:
1. (chain rule of entropy) H(X,Y) = H(X)+H(Y|X). If X and Y are independent,
H(X, Y ) = H(X) + H(Y ).
Proof: Take the expectation of log 1 = log 1 +log 1 and note that 2 pX,Y (X,Y) 2 pX(X) 2 pY|X(Y|X)
PY |X (y|x) = pY (y) for all x, y if X and Y are independent.
2. (mutualinformation)ThemutualinformationsatisfiesI(X;Y)=Y(Y;X)and
  (x,y)?X �Y
pX,Y(x,y)log2 pX(x)pY(y) =E log2 pX(X)pY(Y) .
   I (X; Y ) = H (X) + H (Y ) ? H (X, Y ) = H (X) ? H (X|Y ) = H (Y ) ? H (Y |X) . Proof: Take the expectation of log pX,Y (X,Y ) = log 1 + log 1 ? log 1
    2 pX(X)pY (Y) 2 pX(X) 2 pY (Y) 2 pX,Y (X,Y) and apply the chain rule as needed. Also, symmetry follows from swapping X,Y and
x, y in the sum because pX,Y (x, y) = pY,X (y, x).
Example 23. Let X = Y = {0, 1} and pX,Y (x, y) = ?/2 if x?= y and pX,Y (x, y) = (1 ? ?)/2
ifx=y. SincepX(x)=pY(y)= 1,itfollowsthatH(X)=1andH(Y)=1. Since 2
pY|X(y|x)?{?,1??},itfollowsthatH(Y|X)=H(?). Thus,wehaveI(X;Y)=H(Y)? H (Y |X) = 1 ? H(?). The conditional distribution pY |X called the binary symmetric channel with error probability ? and denoted by BSC(?).
 
7 Definition 24. The Kullback-Liebler (KL) divergence (in bits) between distributions p(x)
x?X
20
ifthereisanyx?X suchthatp(x)>0andq(x)=0.
Remark 25. The divergence is non-negative and equal to 0 iff p(x) = q(x) for all x. Thus, it behaves something like a metric on the space of distributions. It is not exactly a metric, however, because it is not symmetric.
Example 26. For X = {0, 1}, let p(1) = r define a Bernoulli(r) distribution and q(1) = s define a Bernoulli(s) distribution. Then, the divergence between p and q is given by
D(r||s)??rlog r +(1?r)log 1?r. 2s 21?s
Example 27. Let X be the number of ones in a length-n vector of i.i.d. Bernoulli(s) random variables. Then, the probability the resulting vector has exactly rn ones is given by
??n?? rn n?rn P(X=rn)= rn s (1?s) .
Using the results from Example 15, one can see that this equals
and q(x), defined on the same X, is denoted D(p?q)???p(x)log p(x),
 2 q(x) whereweassume0log(0/q)=0forq?[0,1]andplog p =?forp>0. Thus,D(p?q)=?
   1+O(1) 1 1 1+O(1)
nr(1?r) n[rlog +(1?r)log ] n[rlogs+(1?r)log (1?s)] nr(1?r) ?nD(r||s)
  ?? 22r 21?r2 2 =?? 2. 2?nr(1 ? r) 2?nr(1 ? r)
      Thus, we see that exponential decay rate is determined by the divergence between a Bernoulli(r) distribution and a Bernoulli(s) distribution. This example highlights the con- nection between information theory and the theory of large deviations.
Definition 28. A function f : R ? R is called convex on the interval (a, b) if, for all x1,x2 ? (a,b) and ? ? [0,1],
f(?x1 + (1 ? ?)x2) ? ?f(x1) + (1 ? ?)f(x2).
It is called strictly convex if equality holds only if ? = 0 or ? = 1. For a (strictly) convex function f, the function ?f is called (strictly) \textbf{concave}. 

%\begin{defn} 
Let $(\Omega, \sF, P)$ be a probability space. 
For an arbitrary index set $T$ and state space $\sX \subseteq \R$, a \textbf{random process} is a measurable map $X : (\Omega, T) \to \sX$. 
Realization of random process at each $t \in T$, is a random variable defined on the probability space $(\Omega, \sF, P)$ as  $X_t: \Omega \to \sX$ such that  
\EQ{
X_t(\omega) \triangleq X(\omega, t) \in \sX, \text{ and }X_t = (X_t(\omega) \in \sX: \omega \in \Omega).
} 
For each outcome $\omega \in \Omega$, we have a function $X_{\omega}: T \to \sX$ called the \textbf{sample path} or the \textbf{sample function} of the process $X$ defined as 
\EQ{
X_{\omega} \triangleq (X(\omega, t) \in \sX: t \in T) = (X_t(\omega) \in \sX: t \in T).
} 
The random process $X$ can be thought of as a collection of random variables $X = (X_t \in \sX: t \in T)$ or an ensemble of sample paths $X = (X_\omega \in \sX^T: \omega \in \Omega)$. 
Recall that $\sX^T$ is set of all functions from the index set $T$ to state space $\sX$. 

%each defined on the same probability space $(\Omega, \sF, P)$ is called a \textbf{random process} 
%\end{defn}
\begin{shaded*}
\begin{exmp}[Bernoulli sequence] 
\label{exmp:BernoulliSeqDefn}
Let index set $T = \N = \{1, 2, \dots\}$ and the sample space be the collection of infinite bi-variate sequences of successes (S) and failures (F) defined by $\Omega = \{S, F\}^\N$. 
An outcome $\omega \in \Omega$ is an infinite sequence $\omega = (\omega_1, \omega_2, \dots)$ such that $\omega_n \in \set{S, F}$ for each $n \in \N$. 
We define the random process $X: (\Omega, \N) \to \{0,1\}$ such that $X_{\omega} = (\indicator{\omega_1 = S}, \indicator{\omega_2 = S}, \dots)$. 
That is, we have 
\meq{3}{
& X(\omega, n) = \indicator{\omega_n = S}, &&X_n =(\indicator{\omega_n = S}: \omega_n \in \set{S, F}),&&X_\omega = (\indicator{\omega_n = S}: n \in \N).
} 
Hence, we can write the process  as collection of random variables $X = (X_n: n \in \N)$ or the collection of sample paths $X = (X_\omega: \omega \in \Omega)$. 
\end{exmp}
\end{shaded*}

\subsection{Classification}
State space $\sX$ can be countable or uncountable, corresponding to discrete or continuous valued process.  
If the index set $T$ is countable, the stochastic process is called \textbf{discrete}-time stochastic process or random sequence. 
When the index set $T$ is uncountable, it is called \textbf{continuous}-time stochastic process. 
The index set $T$ doesn't have to be time, if the index set is space, and then the stochastic process is spatial process. 
When $T = \R^n \times [0, \infty)$, stochastic process $X$ is a spatio-temporal process. 
\begin{shaded*}
\begin{exmp}
We list some examples of each such stochastic process. 
\begin{enumerate}[i\_]
\item Discrete random sequence: brand switching, discrete time queues, number of people at bank each day.
\item Continuous random sequence: stock prices, currency exchange rates, waiting time in queue of $n$th arrival, workload at arrivals in time sharing computer systems.
\item Discrete random process:  counting processes, population sampled at birth-death instants, number of people in queues.
\item Continuous random process: water level in a dam, waiting time till service in a queue, location of a mobile node in a network.
\end{enumerate}
\end{exmp}
\end{shaded*}

\subsection{Independence}
Recall, given the probability space $(\Omega, \sF, P)$, two events $A, B \in \sF$ are \textbf{independent events} if 
\EQ{
P(A\cap B) = P(A)P(B).
}
Random variables $X,Y$ defined on the above probability space, are \textbf{independent random variables} if for all $x,y \in \R$
\EQ{
P\{X(\omega) \le x, Y(\omega) \le y\} = P\{X(\omega) \le x\}P\{Y(\omega) \le y\}.
}
A stochastic process $X$ is said to be \textbf{independent} if for all finite subsets $S \subseteq T$, the finite collection of events $\set{\set{X_s \le x_s}: s \in S}$ are independent.  
That is, we have
\EQ{
P(\{X_s \le  x_s, s \in S\}) = \prod_{s \in S}P\{X_s \le x_s\}. 
}
Two stochastic process $X, Y$ for the common index set $T$ are \textbf{independent random processes} if for all %$m,n \in \N$ and 
finite subsets $I, J \subseteq T$, 
the following events $\set{X_i \le x_i, i \in I}$ and $\set{Y_j \le y_j, j \in J}$ are independent. %such that $|I| =m, |J| = n$, 
That is, 
\EQ{
P\left(\{X_i \le x_i, i \in I \}\cap\{Y_j \le y_j, j \in J\}\right) = P\left(\{X_i \le x_i, i \in I\}\right)P\left(\{Y_j \le y_j, j \in J\}\right).
}
\begin{shaded*}
\begin{exmp}[Bernoulli sequence] 
\label{exmp:BernoulliSeqIndep}
Let the Bernoulli sequence $X$ defined in Example~\ref{exmp:BernoulliSeqDefn} be independent and identically distributed with $P\set{X_n = 1} = p \in (0,1)$. 
For any sequence $x \in \set{0,1}^\N$, we have $P\set{X = x} = 0$. 
Let $q \triangleq (1-p)$, then the probability of observing $m$ heads and $r$ tails is given by $p^mq^r$. 
\end{exmp}
\end{shaded*}
\subsection{Specification}
To define a measure on a random process, we can either put a measure on sample paths, or equip the collection of random variables with a joint measure. 
%We adopt the second approach, and 
We are interested in identifying the joint distribution $F: \R^T \to [0,1]$. 
To this end, for any $x \in \R^T$we need to know
\EQ{
F(x) = P\left(\displaystyle {\bigcap_{t \in T}\{\omega \in \Omega: X_t(\omega) \le x_t\}}\right) = P(\bigcap_{t \in T}X_t^{-1}(-\infty, x_t]) = P \circ X^{-1}\bigtimes_{t \in T}(-\infty, x_t].
}
However, even for a simple independent process with countably infinite $T$, 
%When the index set $T$ is infinite, 
any function of the above form would be zero if $x_t$ is finite for all $t \in T$. 
Therefore, we only look at the values of $F(x)$ when $x_t \in \R$ for indices $t$ in a finite set $S$ and $x_t = \infty$  for all $t \notin S$. 
That is, for any finite set $S \subseteq T $we focus on the product sets of the form 
\EQ{
\bigtimes_{s \in S}(-\infty, x_s]\bigtimes_{s \notin S}\R. 
}
%This leads to finite-dimensional distributions as defined below. 
%\begin{defn}  
We can define a \textbf{finite dimensional distribution} for any finite set $S \subseteq T$ and $x_S = \{x_s \in \R : s \in S\}$, 
\EQ{
F_S(x_S) = P\left(\displaystyle {\bigcap_{s \in S}\{\omega \in \Omega: X_s(\omega) \le x_s\}}\right) = P(\bigcap_{s \in S}X_s^{-1}(-\infty, x_s]).
}
%\end{defn}
Set of all finite dimensional distributions of the stochastic process $\{X_t: t \in T\}$ characterizes its distribution completely.
Simpler characterizations of a stochastic process $X(t)$ are in terms of its moments. 
That is, the first moment such as mean, and the second moment such as correlations and covariance functions. 
\meq{3}{
&m_X(t) \triangleq \E X_t, && R_X(t,s) \triangleq \E X_tX_s,&& C_X(t,s) \triangleq \E (X_t - m_X(t))(X_s-m_X(s)).
}

\begin{shaded*}
\begin{exmp}[Bernoulli sequence] 
We can easily compute the mean, the auto-correlation, and the auto-covariance functions for the independent Bernoulli process defined in Example~\ref{exmp:BernoulliSeqIndep} as 
\meq{3}{
&m_X(n) = \E X_n = p, &&R_X(m,n)=\E X_mX_n = \E X_m\E X_n = p^2,&&C_x(m,n) = 0.
}
\end{exmp}
\end{shaded*}

\begin{shaded*}
\begin{exmp}
Some examples of simple stochastic processes. 
\begin{enumerate}[i\_]
\item $X_t = A \cos 2\pi t$, where $A$ is random. 
%The finite dimensional distribution is given by 
%\begin{align*}
%F_S(x) = P\left(\left\{A\cos 2\pi s \leq x_s, s \in S\right\}\right). %P\left(\left\{A \leq \min_{s \in S\setminus\{(2k+1)\frac{\pi}{2}, k \in \Z\}}\frac{x_s}{\cos 2\pi_s}\right\}\right).
%\end{align*}
%The moments are given by 
%\meq{3}{
%&m_X(t) = (\E A)\cos 2\pi t, && R_X(t,s) =  (\E A^2) \cos 2\pi t\cos 2\pi s,&& C_X(t,s) =\text{Var}(A) \cos 2\pi t\cos 2\pi s.
%}
\item $X_t = \cos(2\pi t+ \Theta)$, where $\Theta$ is random and uniformly distributed between $(-\pi, \pi]$. 
%The finite dimensional distribution is given by 
%\begin{align*}
%F_S(x) = P\left(\left\{\cos(2\pi s + \Theta) \leq x_s, s \in S\right\}\right). %P\left(\left\{A \leq \min_{s \in S\setminus\{(2k+1)\frac{\pi}{2}, k \in \Z\}}\frac{x_s}{\cos 2\pi_s}\right\}\right).
%\end{align*}
%The moments are given by 
%\meq{3}{
%&m_X = 0, && R_X(t,s) =  \frac{1}{2}\cos2\pi (t-s),&& C_X(t,s) = R_X(t,s).
%}
\item $X_n = U^n$ for $n \in \N$, where $U$ is uniformly distributed in the open interval $(0,1)$.
\item $Z_t = At +B$ where $A$ and $B$ are independent random variables. 
\end{enumerate}
\end{exmp}
\end{shaded*}

\end{document}

\section{Examples of Tractable Stochastic Processes}
In general, it is very difficult to characterize a stochastic process completely in terms of its finite dimensional distribution. 
However, we have listed few analytically tractable examples below, where we can completely characterize the stochastic process. 
\subsection{Independent and identically distributed processes}
Let $(X_t: t \in T)$ be an independent and identically distributed (\emph{iid}) random process, with a common distribution $F(x)$. 
Then, the finite dimensional distribution for this process for any finite $S \subseteq T$ can be written as 
\EQ{
F_S(x_S) = P\left(\{X_s(\omega) \le x_s, s \in S\}\right) = \prod_{s \in S}F(x_s).
}
It's easy to verify that the first and the second moments are independent of time indices. 
Since $X_t = X_0$ in distribution, 
\meq{3}{
&m_X = \E X_0 , && R_X = \E X_0^2, && C_X = \Var(X_0).
}

%\subsubsection{Independent increments process}
\subsection{Stationary processes}
A stochastic process $X$ is \textbf{stationary} if all finite dimensional distributions are shift invariant. 
That is, for any finite $n \in \N$ and $t > 0$, the random vectors $(X_{s_1}, \dots, X_{s_n})$ and $(X_{s_1+t}, \dots, X_{s_1+t})$ have the identical joint distribution for all $s_1 \le \dots \le s_n$.  
That is, for finite $S \subseteq T$ and $t > 0$, we have 
\EQ{
F_S(x_S) = P(\{X_s(\omega) \le x_s, s \in S\}) = P(\{X_{s+t}(\omega) \le x_s, s \in S\}) = F_{t+S}(x_S).
}
For a stationary stochastic process, all the existing moments are shift invariant. 
For Gaussian random processes, first and the second moment suffice to get any finite dimensional distribution. 
A \textbf{second order} stochastic process $X$ has finite auto-correlation $R_X(t,t) < \infty$ for all indices $t \in T$. 
This implies $R_X(t_1, t_2) < \infty$ by Cauchy-Schwartz inequality, 
and hence the mean, auto-correlation, and the auto-covariance functions are well defined and finite. 
Since $X_t = X_0$ and $(X_t,X_s) = (X_{t-s}, X_{0})$ in distribution, we have for a second order process $X$ 
\meq{3}{
&m_X = \E X_0 , && R_X(t, s) = R_X(t-s,0) = \E X_{t-s}X_0, && C_X(t-s, 0) = R_X(t-s, 0) - m_X^2.
}


A random process $X$ is \textbf{wide sense stationary} if 
\begin{enumerate}
\item $m_X(t) = m_X(t+s)$ for all $s,t \in T$, and 
\item $R_X(t,s) = R_x(t+u, s+u)$ for all $s,t,u \in T$. 
\end{enumerate} 
It follows that a second order stationary stochastic process $X$,  is wide sense stationary. 
A second order wide sense stationary process is not necessarily stationary. 
We can similarly define join stationarity and joint wide sense stationarity for two stochastic processes $X$ and $Y$.

\subsection{Markov processes}
A stochastic process $X$ is \textbf{Markov} if conditioned on the present state, future is independent of the past. 
We denote the history of the process until time $t$ as $\sF_t = \sigma(X_s, s \le t)$. 
That is, for any ordered index set $T$ containing any two indices $u > t$, we have  %such that $t < u$, % \in T$, 
\EQ{
P(\{X_u \le x_u\}\given \sF_t) = P(\{X_{u} \le x_u\}\given \sigma(X_t)).
}
The range of the process is called the \textbf{state space}. 
%For a stationary Markov process, we would have 
We next re-write the Markov property more explicitly for the process $X$. 
For all $x, y \in \sX$, finite set $S \subseteq T$ such that $\max S < t < u$, and $H_{s}  = \set{X_s \le x_s: s \in S} \in \sF_t$, we have 
\EQ{
P(\{X_u \le y\}\given H_S \cap \set{X_t \le x}) = P(\{X_{u} \le y\}\given \set{X_t \le x}).
}
When the state space $\sX$ is countable, we can write $H_S = \cap_{s \in S}\set{X_s = x_s}$ and the Markov property can be written as 
\EQ{
P(\{X_u = y\}\given H_S \cap \set{X_t = x}) = P(\{X_{u} = x_u\}\given \set{X_t = x}).
}
We will study this process in detail in coming lectures. 


\begin{shaded*}
\begin{exmp}[Random Walk] 
Let $X = (X_n \in \R^d: n \in \N)$ be an be an independent (not necessarily identical) Bernoulli sequence. 
Let $S_0 = 0$ and $S_n \triangleq \sum_{i=1}^nX_i$, then the process $S = (S_n \in \R^d: n \in \N_0)$ is called a \textbf{random walk}. 
We can think of $S_n$ as the random location of a particle after $n$ steps, 
where the particle starts from origin and takes steps of size $X_i$ at the $i$th step. 

%From previous section, we know following properties of random walks. 
\begin{thm} 
For a random walk $(S_n: n \in \N)$ with independent step-size sequence $X$, the following are true. 
\begin{enumerate}[i\_]
\item The first two moments are $\E S_n = \sum_{i=1}^n\E X_i$ and $\Var[S_n] = \sum_{i=1}^n\Var[X_i]$. 
\item Random walk is non-stationary with independent increments. 
\item Random walk is a Markov sequence. 
\end{enumerate}
\end{thm}
\proof{
Results follow from the independence of the step-size sequence $X$. 
\begin{enumerate}[i\_]
\item Follows from the linearity of expectation and independence of step sizes. 
\item Since the mean is time dependent, random walk is non-stationary process. 
Independence of increments follows from the independence of step sizes. 
\item Given the historical event $H_{n-1} \triangleq \cap_{k=1}^{n-1}\set{S_k \le s_k}$ and the current state $\set{S_n \le s_n}$, 
we can write the conditional probability 
\eq{
P(\set{S_{n+1} \le s_{n+1}}\given H_{n-1}\cap\set{S_n \le s_n}) &= P(\set{X_{n+1} \le s_{n+1}-S_n}\given H_{n-1}\cap\set{S_n \le s_n})\\
&= P(\set{S_{n+1} \le s_{n+1}}\given \set{S_n \le s_n}). % = P\set{X_{n+1} = s_{n+1}-s_n} .
}
The equality in the second line follows from the independence of the step-size sequence. 
In particular, from the independence of $X_{n+1}$ from the collection $\sigma(S_0, X_1, \dots, X_n) = \sigma(S_0, S_1, \dots, S_n)$. 
For the countable state space $\sX$, an given the historical event $H_{n-1} \triangleq \cap_{k=1}^{n-1}\set{S_k = s_k}$ and the current state $\set{S_n = s_n}$, 
we can write the conditional probability 
\eq{
P(\set{S_{n+1} = s_{n+1}}\given H_{n-1}\cap\set{S_n = s_n}) &= P(\set{X_{n+1} = s_{n+1}-S_n}\given H_{n-1}\cap\set{S_n = s_n})\\
&= P(\set{S_{n+1} = s_{n+1}}\given \set{S_n = s_n}) = P\set{X_{n+1} = s_{n+1}-s_n} .
}
\end{enumerate}
}
\end{exmp}
\end{shaded*}

\subsection{L\'evy processes}
A right continuous with left limits stochastic process $X = (X_t \in \R: t \in T \subseteq \R_+)$ with $X_0 = 0$ almost surely, is a \textbf{L\'evy process} if the following conditions hold. 
\begin{enumerate}[(L1)]
%\item $X_{0}=0$ almost surely.
\item The increments are independent. For any $ 0\leq t_{1}<t_{2}<\cdots <t_{n}<\infty$, $X_{t_{2}}-X_{t_{1}},X_{t_{3}}-X_{t_{2}},\ldots ,X_{t_{n}}-X_{t_{n-1}}$ are independent.
\item The increments are stationary. For any $s<t$,  $X_{t}-X_{s}$, is equal in distribution to $X_{t-s}$.
\item Continuous in probability. For any $\epsilon > 0$ and $t\geq 0$ it holds that $\lim _{h\rightarrow 0}P(|X_{t+h}-X_{t}|>\epsilon )=0$. 
\end{enumerate}

\begin{shaded*}
\begin{exmp}
Two examples of L\'evy processes are Poisson process and Wiener process. 
The distribution of Poisson process at time $t$ is Poisson with rate $\lambda t$ and the distribution of Wiener process at time $t$ is zero mean Gaussian with variance $t$. 
\end{exmp}
\end{shaded*}

\section{Conditional Expectation} 
Let $(\Omega, \sF, P)$ be the probability space.  
Let $X$ be a measurable random variable on this probability space denoted as $X \in \sF$, 
if the event $X^{-1}(-\infty, x] = \{\omega \in \Omega: X(\omega) \leq x\} \in \sF$ for each $x \in \R$.  
Let $\sE \subseteq \sF$ be a $\sigma$-algebra, then the \textbf{conditional expectation} of $X$ given $\sE$ is denoted $\E[X | \sE]$ and is a random variable $Y = \E[X | \sE]$ where
\begin{enumerate}[i\_] 
\item $Y \in \sE$,
\item for each event $A \in \sE$, we have $\E [X1_A] = \E[Y1_A]$. 
%\eq{
%\int_A X dP = \int_A Y dP. 
%}
\end{enumerate}
Intuitively, we think of the $\sigma$-algebra $\sE$ as describing the information we have. 
For each $A \in \sE$, we know whether or not $A$ has occurred. 
The conditional expectation $\E[X|\sE]$ is then the ``best guess'' of the value of $X$ given the information $\sE$. 
Let $X,Y$ be two random variables defined on this probability space. 
Then, the conditional expectation of $X$ given $Y$ is defined as 
\eq{
\E[X|Y] = \E[X | \sigma(Y)]. 
}
A random variable $X$ is \textbf{independent} of the $\sigma$-algebra $\sE$, if for all $x \in \R$ and $A \in \sE$, 
\eq{
\E[1_{\{X \leq x\}}1_A]= P\{X \leq x\}\cap A = P\{X \leq x\}P(A) = \E1_{\{X \leq x\}}\E1_A. 
}
\begin{lem}
Let $(\Omega, \sF, P)$ be a probability space with $\sE \subseteq \sF$ a $\sigma$-algebra. 
If $X \in \sE$ is a random variable, then $\E[X| \sE] = X$. 
\end{lem}
\begin{proof} 
First condition is true by hypothesis, and the second condition holds for any $A \in \sE$. 
\end{proof}
\begin{lem} 
Let $(\Omega, \sF, P)$ be a probability space with $\sE \subseteq \sF$ a $\sigma$-algebra. 
If $X \in \sF$ be a random variable independent of $\sE$.  
Then, $\E[X| \sE]  = \E[X]$. 
\end{lem}
\begin{proof}
This follows since $\E X \in \sE$ and the random variables $X$ and $1_{A}$ are independent for any $A \in \sE$, 
which implies 
\eq{
\E[X1_A] = \E X \E 1_A = \E [(\E X) 1_A]. 
}
\end{proof}
One can partition the state space $\R$ into measurable sets $E_1, E_2, \dots$ for the random variable $X$ defined on the given probability space. 
Then $\Omega_i \triangleq X^{-1}(E_i)$ is a partition of the sample space $\Omega$. 
Let $Y$ be a random variable defined as the partition index for the random variable $X$. 
That is, 
\eq{
Y = \sum_{i \in \N}i \cdot 1_{\{X \in E_i\}}. 
} 
Let $\sE \triangleq \sigma(\Omega_1, \Omega_2, \dots)$, then one can check that $Y \in \sE$ or $\sigma(Y) = \sE$. 
Hence, $\E[X|Y] = \E[X|\sigma(Y)] = \E[X|\sE]$. 
Clearly, $\E[X|Y]$ would be a function of $Y$ and since $Y$ takes countably many values, we have $Z = \E[X|Y]$ taking countably many values, with $Z_i = Z1_{\{Y = i\}}$ being a constant on the corresponding partition $\Omega_i$  of the sample space. 
One can compute this conditional expectation using joint distribution directly as 
\eq{
\E[X|Y=i] = \int_{\R}xdF_{X|Y=i}(x) = \frac{1}{P(\Omega_i)}\int_{E_i}xdF(x) = \frac{\E X1_{\Omega_i}}{P(\Omega_i)}
}
\begin{lem}
Suppose $\{\Omega_i: i \in \N\}$ be a countable partition of the sample space $\Omega$, and $\sE = \sigma(\Omega_1, \Omega_2, \dots)$ is the $\sigma$-field generated by this partition. 
Then, 
\eq{
\E[X | \sE] = \frac{\E[X1_{\Omega_i}]}{P(\Omega_i)} \text{ on } \Omega_i.
}
\end{lem}
\begin{proof}
It is easy to see that the RHS is constant on each partition $\Omega_i$ and hence is measurable with respect to $\sE$. 
Further, for each $\Omega_i \in \sE$, we have
\eq{
\int_{\Omega_i}\frac{\E[X 1_{\Omega_i}]}{P(\Omega_i)}dP = \E[X 1_{\Omega_i}] = \int_{\Omega_i}XdP. 
}
\end{proof}
\begin{cor} $P(A|B)P(B) = P(A\cap B)$.
\end{cor}
\begin{proof}
Taking $X = 1_A$ and $\sE = \{\emptyset, \Omega, B, B^c\}$, from the previous Lemma we get 
\eq{
P(A|B) = \E[1_A| 1_B]= \E[1_A|\sE] = \frac{\E[1_A1_B]}{P(B)} = \frac{P(A\cap B)}{P(B)}.
} 
\end{proof}

\begin{thm}[Bayes' Formula] 
For a $\sigma$-algebra $\sE \subseteq \sF$, and for any events $G \in \sE$ and $A \in \sF$, we have 
\eq{
P(G|A) = \frac{\E[1_GP(A| \sE)]}{\E P(A|\sE)}.
}
\end{thm}
\begin{proof}
It is easy to check that numerator is $\E 1_G\E[1_A|\sE] = \E[1_{A \cap G}|\sE]$. 
It suffices to show that $\E\E[1_A|\sE] = \E 1_A$, which follows from definition. 
\end{proof}
\begin{cor} 
For the countable partition $(\Omega_1, \Omega_2, \dots)$, if the $\sigma$-algebra $\sE = \sigma(\Omega_1, \Omega_2, \dots)$, then for any events $G \in \sE$ and $A \in \sF$, we have 
\eq{
P(\Omega_i|A) = \frac{P(A|\Omega_i)P(\Omega_i)}{\sum_{j \in \N}P(A|\Omega_j)P(\Omega_j)}.
}
\end{cor}
\begin{proof}
Result follows from the fact that $P(A|\sE) \in \sE$ and hence is a constant on each partition $\Omega_j$. % and $\sum_{j \in \N}1_{\Omega_j} = 1$.
\end{proof}

\subsection{Filtration}
A net of $\sigma$-algebras $\sF_{\bullet} = \{\sF_t \subseteq \sF: t \in T\}$ is called a \textbf{filtration} when the index set $T$ is totally ordered and the net is non-decreasing, that is for all $s \leqslant t \in T$ implies $\sF_s \subseteq \sF_t$. 
Consider a random process $X$ indexed by the ordered set $T$ on the probability space $(\Omega, \sF, P)$. 
The process $X$ is called \textbf{adapted} to the filtration $\sF_{\bullet}$, if for each $t \in T$, we have the random variable $X_t \in \sF_t$. 
For a random process $X$ with an ordered index set $T$, we can define a natural filtration $\sF_{\bullet} = \{\sF_t \subseteq \sF: t \in T\}$ indexed by $T$, where  $\sF_t \triangleq \sigma(X_s, s \leqslant t)$ is the information about the process till index $t$ and 
the process $X$ is adapted to its natural filtration by definition. 

If $X = (X_t: t \in T)$ is an independent process with the associated natural filtration $\sF_{\bullet}$, then for any $t > s$ and events $A \in \sF_s$, 
$X_t$ is independent of the event $A$. 
This is just a fancy way of saying $X_t$ is independent of $(X_u, u \le s)$. 
Hence, for any random variable $Y \in \sF_s$, we have 
\eq{
\E[\E[X_tY|\sF_s]] = \E[\E[X_t] Y] = \E X_t \E Y.
}

\end{document}